\documentclass[11pt]{article}

\usepackage{geometry}
\geometry{
  a4paper,         % or letterpaper
  textwidth=16cm,  % llncs has 12.2cm
  textheight=24cm, % llncs has 19.3cm
  heightrounded,   % integer number of lines
  hratio=1:1,      % horizontally centered
  vratio=2:3,      % not vertically centered
}

\renewcommand\labelitemi{*}

% Packages
% ------------------
\usepackage{tkz-euclide}
\usepackage{tikz} % for drawing diagrams
\usepackage{bm} % bold mode
\usepackage{subfig}
\usepackage{amssymb}
\usepackage{amsmath}
\usepackage{color}
\usepackage{dsfont}
\usepackage{forest}
% \usepackage{algorithm}
% \usepackage{algpseudocode}
% \usepackage{graphicx} % for the \includegraphics command
% \usepackage{titlesec} % allow for changing the spacing in sections


% setup style of tikzlibrary elements
% ------------------------------------
\usetikzlibrary{arrows}
\usetikzlibrary{quotes,angles}
\usetikzlibrary{positioning}
\usetikzlibrary{plotmarks}
\usetikzlibrary{shapes.geometric, arrows}
\tikzstyle{block} = [rectangle, minimum width=2.5cm, minimum height=1cm, text centered, text width=2.7cm, draw=black, fill=white]
\tikzstyle{redblock} = [rectangle, thick, minimum width=2.5cm, minimum height=1cm, text centered, text width=2.7cm, draw=red, fill=white]
\tikzstyle{wideblock} = [rectangle, minimum width=2.5cm, minimum height=1cm, text centered, text width=5cm, draw=black, fill=white]
\tikzstyle{tallblock} = [rectangle, minimum width=1.9cm, minimum height=2cm, text centered, text width=1.9cm, draw=black, fill=white]
\tikzstyle{inputblock} = [rectangle, minimum width=1.7cm, minimum height=5cm, text centered, text width=1.7cm, draw=black, fill=white]
\tikzstyle{startstop} = [rectangle, rounded corners, minimum width=3cm, minimum height=1cm, text centered, draw=black, fill=white]
\tikzstyle{startstop_wide} = [rectangle, rounded corners, minimum width=6cm, minimum height=1cm, text centered, draw=black, fill=white]
\tikzstyle{io} = [trapezium, trapezium left angle=70, trapezium right angle=110, minimum width=3cm, minimum height=1cm, text centered, draw=black, fill=white]
\tikzstyle{process} = [rectangle, minimum width=4cm, minimum height=1cm, text centered, text width=4cm, draw=black, fill=white]
\tikzstyle{decision} = [diamond, minimum width=3cm, minimum height=1cm, text centered, draw=black, fill=white]
\tikzstyle{arrow} = [->,>=stealth] % can add "thick" to make it bold
\tikzset{radiation/.style={{decorate,decoration={expanding waves,angle=90,segment length=4pt}}}}

%% new commands
% ------------------
\newcommand{\Z}{\mathbb{Z}}
\newcommand{\norm}[1]{\lVert#1\rVert}
\newcommand{\abs}[1]{\left|#1\right|}
\newcommand*\Let[2]{\State #1 $\gets$ #2}

\definecolor{light-gray}{gray}{0.95}
\newcommand{\code}[1]{\colorbox{light-gray}{\texttt{#1}}}

\begin{document}

% {\Huge Project Description}

% Header of paper
% Hint: \title{what ever}, \author{who care} and \date{when ever} could stand
% before or after the \begin{document} command
% BUT the \maketitle command MUST come AFTER the \begin{document} command!
% ---------------------
\title{TikZ Library Template}
\author{
  Sergio Garc\'{i}a-Vergara \\
  RIF Robotics Corp \\
}

\maketitle

%@@@@@@@@@@@@@@@@@@@@@@@@@@@@@@@@@@@@@@@@@@@@@@@@@@


% % Abstract
% % ---------------------
% \begin{abstract}
%   Maybe this would be good to have.

% \begin{keywords}
%   one $\cdot$ two
% \end{keywords}

% \end{abstract}


% @@@@@@@@@@@@@@@@@@@@@@@@@@@@@@@@@@@@@@@@@@@@@@@@@@

\section{System Diagram}
% ---------------------------

This diagram makes no sense as right now I'm using it as a template for real
diagrams. I'll be sure to clean it up and link it to better explanations on how
to create these diagrams.

\vspace{1em}
\noindent NOTES:
\begin{itemize}
  \item Here you would add notes
\end{itemize}


% diagram - short term
\begin{figure}[h]
\begin{center}
\begin{tikzpicture}[auto, node distance=2.7cm, >=latex']

  % We start by placing the blocks
  \node (camera) [block] {Camera};
  \node (block1) [block, right of=camera, xshift=0.8cm] {Block1};
  \node (level1) [block, right of=block1, xshift=1.6cm] {Level1};
  \node (redblock) [redblock, below of=level1, yshift=0.6cm] {Red Block};
  \node (database) [redblock, below of=redblock, yshift=0.6cm] {Database};

  % main box
  \node (main_box) [rectangle, dashed, minimum width=6cm, minimum height=6.4cm,
    text width=2.7cm, draw=black, fill=white, fill opacity=0.1, right of=block1,
    xshift=1.8cm, yshift=-2cm] {};
  \node (main_box_name) [above of=main_box, xshift=-1.4cm, yshift=0.8cm] {Main Box};

  \node (gimbal) [block, below of=main_box, yshift=-2cm] {Gimbal};
  \node (radio) [block, right of=main_box, xshift=2.5cm] {Radio};
  \node (air) [right of=radio, xshift=-0.3cm] {};

  % draw connections between nodes
  \draw [arrow] (camera) -- node[anchor=north] {} (block1);
  \draw [arrow] (block1) -- node[anchor=north] {Level 1} (level1);
  \draw [arrow] (level1) -- node[anchor=east] {Level 2} (redblock);
  \draw [arrow] (redblock) -- node[anchor=east] {Level 3} (database);
  \draw [<->] (main_box) -- node[anchor=east] {serial commands} (gimbal);
  \draw [radiation,decoration={angle=45}] (radio) -- (air);
  \draw [<->] (main_box) -- node[anchor=north] {} (radio);
  \draw [arrow] (main_box.0) -| ++(-1cm,0) |- node[anchor=south, xshift=0.4cm] {tracks} (level1.0);

\end{tikzpicture}
\end{center}
 \caption{Short term system diagram.}
\label{fig:short_term_system_diagram}
\end{figure}

\newpage

%@@@@@@@@@@@@@@@@@@@@@@@@@@@@@@@@@@@@@@@@@@@@@@@@@@


\section{file diagram}
% ---------------------------

% See https://tex.stackexchange.com/questions/5073/making-a-simple-directory-tree

\begin{forest}
  for tree={
    font=\ttfamily,
    grow'=0,
    child anchor=west,
    parent anchor=south,
    anchor=west,
    calign=first,
    edge path={
      \noexpand\path [draw, \forestoption{edge}]
      (!u.south west) +(7.5pt,0) |- node[fill,inner sep=1.25pt] {} (.child anchor)\forestoption{edge label};
    },
    before typesetting nodes={
      if n=1
      {insert before={[,phantom]}}
      {}
    },
    fit=band,
    before computing xy={l=15pt},
  }
  [projects
    [Main Proj
      [Sub1
        [filename]
        [UAV17
          [0017.parm]
          [Pilots
            [pilot1]
            [pilot2]
          ]
        ]
        [UAV43
          [filename]
          [Pilots
            [...]
          ]
        ]
      ]
      [Sub2
        [...]
      ]
    ]
    [Secondary Project
      [...]
    ]
  ]
\end{forest}


% @@@@@@@@@@@@@@@@@@@@@@@@@@@@@@@@@@@@@@@@@@@@@@@@@@

\section{hello world}
% ---------------------------


% @@@@@@@@@@@@@@@@@@@@@@@@@@@@@@@@@@@@@@@@@@@@@@@@@@

\newpage

\bibliographystyle{ieeetr}
\bibliography{references}

\end{document}

%%% Local Variables:
%%% mode: latex
%%% TeX-master: t
%%% End:
