\documentclass[11pt]{article}

\usepackage{geometry}
\geometry{
  a4paper,         % or letterpaper
  textwidth=16cm,  % llncs has 12.2cm
  textheight=24cm, % llncs has 19.3cm
  heightrounded,   % integer number of lines
  hratio=1:1,      % horizontally centered
  vratio=2:3,      % not vertically centered
}

\renewcommand\labelitemi{*}

% Packages
% ------------------
\usepackage{tkz-euclide}
\usepackage{tikz} % for drawing diagrams
\usepackage{bm} % bold mode
%% \usepackage{color}
\usepackage{hyperref}


% setup style of tikzlibrary elements
% ------------------------------------
\usetikzlibrary{arrows}
\usetikzlibrary{quotes,angles}
\usetikzlibrary{positioning}
\usetikzlibrary{plotmarks}
\usetikzlibrary{shapes.geometric, arrows}
\tikzstyle{sumation} = [circle, minimum width=0.2cm, minimum height=0.5cm, text centered, text width=0.4cm, draw=black, fill=white]
\tikzstyle{block} = [rectangle, minimum width=2.5cm, minimum height=1cm, text centered, text width=2.7cm, draw=black, fill=white]
\tikzstyle{redblock} = [rectangle, thick, minimum width=2.5cm, minimum height=1cm, text centered, text width=2.7cm, draw=red, fill=white]
\tikzstyle{wideblock} = [rectangle, minimum width=2.5cm, minimum height=1cm, text centered, text width=5cm, draw=black, fill=white]
\tikzstyle{tallblock} = [rectangle, minimum width=1.9cm, minimum height=2cm, text centered, text width=1.9cm, draw=black, fill=white]
\tikzstyle{inputblock} = [rectangle, minimum width=1.7cm, minimum height=5cm, text centered, text width=1.7cm, draw=black, fill=white]
\tikzstyle{startstop} = [rectangle, rounded corners, minimum width=3cm, minimum height=1cm, text centered, draw=black, fill=white]
\tikzstyle{io} = [trapezium, trapezium left angle=70, trapezium right angle=110, minimum width=3cm, minimum height=1cm, text centered, draw=black, fill=white]
\tikzstyle{process} = [rectangle, minimum width=4cm, minimum height=1cm, text centered, text width=4cm, draw=black, fill=white]
\tikzstyle{decision} = [diamond, minimum width=3cm, minimum height=1cm, text centered, draw=black, fill=white]
\tikzstyle{arrow} = [->,>=stealth] % can add "thick" to make it bold
\tikzset{radiation/.style={{decorate,decoration={expanding waves,angle=90,segment length=4pt}}}}

% new commands
% ------------------
\definecolor{light-gray}{gray}{0.95}
\newcommand{\code}[1]{\colorbox{light-gray}{\texttt{#1}}}


\begin{document}

% {\Huge Project Description} % optional

% Header of paper
% NOTE: The \maketitle command MUST come AFTER the \begin{document} command!
% ---------------------
\title{TikZ Playground}
\author{
  Author: Sergio Garc\'{i}a-Vergara \\
  \small{https://www.sergiogarciavergara.com}
}

\maketitle

%@@@@@@@@@@@@@@@@@@@@@@@@@@@@@@@@@@@@@@@@@@@@@@@@@@

% Abstract
% ---------------------
\begin{abstract}
  This sample tex file is intended to be used together with the tutorials on the
  TikZ package described in Sergio's website
  (\href{https://www.sergiogarciavergara.com/block-diagrams-in-latex/}{sergiogarciavergara.com}). The
  tutorials describe step-by-step how to construct the diagrams and flowcharts
  shown in this document.

\end{abstract}


% @@@@@@@@@@@@@@@@@@@@@@@@@@@@@@@@@@@@@@@@@@@@@@@@@@

\section{Block Diagrams}
% ---------------------------

\subsection{Simple Feedback Diagram}

Consider the transfer function for a noise-free physical linear system
(\ref{eq:linear_system_trans_fun}):

\begin{equation}\label{eq:linear_system_trans_fun}
Y(s) = \frac{C(s)}{R(s)} = \frac{KG(s)}{1 + KG(s)H(s)}
\end{equation}

\noindent
where $C(s)$ and $R(s)$ are the output and input of the system,
respectively. It's corresponding block diagram is represented by Figure
\ref{fig:liner_system_diagram}.

\begin{figure}[h]
\begin{center}
\begin{tikzpicture}[auto, node distance=2.7cm, >=latex']

  % We start by placing the blocks
  \node (input) [circle] {R(s)};

  \node (sum) [sumation, right of=input, xshift=-0.4cm] {};
  \node (plus_sign) [above of=sum, xshift=-0.6cm, yshift=-2.4cm] {+};
  \node (minus_sign) [below of=sum, xshift=-0.4cm, yshift=2.1cm] {-};
  \node (kgs) [block, right of=sum, xshift=0.45cm] {KG(s)};
  \node (hs) [block, below of=kgs, yshift=0.7cm] {H(s)};
  \node (output) [circle, right of=kgs, xshift=1.2cm] {C(s)};

  % Draw the connections between blocks
  \draw [arrow] (input) -- node[anchor=north] {} (sum);
  \draw [arrow] (sum) -- node[anchor=north] {} (kgs);
  \draw [arrow] (kgs) -- node[anchor=north] {} (output);
  \draw [arrow] (kgs.0) -| ++(1cm,0cm) |- node[anchor=north] {} (hs.0);
  \draw [arrow] (hs.180) -| node[anchor=north] {} (sum.270);
  % \draw [arrow] (main_box.0) -| ++(-1cm,0) |- node[anchor=south, xshift=0.4cm] {tracks} (level1.0);

\end{tikzpicture}
\end{center}
 \caption{Block diagram of a noise-free linear system.}
\label{fig:liner_system_diagram}
\end{figure}


\subsection{Complicated Block Diagram}

\newpage

%@@@@@@@@@@@@@@@@@@@@@@@@@@@@@@@@@@@@@@@@@@@@@@@@@@

\section{Flowchart}
% ---------------------------

Include examples of flowcharts.

\begin{figure}[h]
\begin{center}
\begin{tikzpicture}[auto, node distance=2.7cm, >=latex']

  \node (main_box) [rectangle, dashed, minimum width=6cm, minimum height=6.4cm,
    text width=2.7cm, draw=black, fill=white, fill opacity=0.1,
    xshift=1.8cm, yshift=-2cm] {};

\end{tikzpicture}
\end{center}
 \caption{Flowchart example.}
\label{fig:flowchart}
\end{figure}

\newpage

%@@@@@@@@@@@@@@@@@@@@@@@@@@@@@@@@@@@@@@@@@@@@@@@@@@

\section{Plots}
% ---------------------------

Include examples of plots.


% @@@@@@@@@@@@@@@@@@@@@@@@@@@@@@@@@@@@@@@@@@@@@@@@@@

% \newpage
% \bibliographystyle{ieeetr}
% \bibliography{references}

\end{document}

%%% Local Variables:
%%% mode: latex
%%% TeX-master: t
%%% End:
